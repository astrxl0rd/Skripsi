\documentclass[a4paper,twoside]{article}
\usepackage[T1]{fontenc}
\usepackage[bahasa]{babel}
\usepackage{graphicx}
\usepackage{graphics}
\usepackage{float}
\usepackage[cm]{fullpage}
\pagestyle{myheadings}
\usepackage{etoolbox}
\usepackage{setspace} 
\usepackage{lipsum} 
\setlength{\headsep}{30pt}
\usepackage[inner=2cm,outer=2.5cm,top=2.5cm,bottom=2cm]{geometry} %margin
% \pagestyle{empty}

\makeatletter
\renewcommand{\@maketitle} {\begin{center} {\LARGE \textbf{ \textsc{\@title}} \par} \bigskip {\large \textbf{\textsc{\@author}} }\end{center} }
\renewcommand{\thispagestyle}[1]{}
\markright{\textbf{\textsc{AIF401/AIF402 \textemdash Rencana Kerja Skripsi \textemdash Sem. Ganjil 2020/2021}}}

\newcommand{\HRule}{\rule{\linewidth}{0.4mm}}
\renewcommand{\baselinestretch}{1}
\setlength{\parindent}{0 pt}
\setlength{\parskip}{6 pt}

\onehalfspacing
 
\begin{document}

\title{\@judultopik}
\author{\nama \textendash \@npm} 

%tulis nama dan NPM anda di sini:
\newcommand{\nama}{Reynaldi Rahadian}
\newcommand{\@npm}{2016730028}
\newcommand{\@judultopik}{Pembuatan Sistem Manajemen Dokumen Berbasis Web} % Judul/topik anda
\newcommand{\jumpemb}{1} % Jumlah pembimbing, 1 atau 2
\newcommand{\tanggal}{01/01/1900}

% Dokumen hasil template ini harus dicetak bolak-balik !!!!

\maketitle

\pagenumbering{arabic}

\section{Deskripsi}

Tidak jarang seseorang memiliki banyak dokumen dalam bentuk softcopy. Dokumen tersebut dapat berupa Jurnal, E-book, Poster, dll. Penamaan dokumen tidak jarang belum terstruktur dan hanya dinamakan sekedarnya berdasarkan judul atau topik tertentu. Kesulitan akan dirasakan jika seseorang tersebut ingin mencari dokumen yang berisi informasi yang sedang dibutuhkan. Jika ada beberapa versi dokumen, seseorang kadang kesulitan untuk menentukan dokumen mana yang merupakan versi lebih terbaru. Aplikasi berbasis web memudahkan pengaksesan oleh penggunam karena tidak diperlukan instalasi program tambahan selain browser. Aplikasi berbasi cloud seperti google drive belum mendukung fitur pencarian selain dengan nama file. Pembuatan aplikasi manajemen dokumen berbasis web akan memudahkan pengguna jika suatu saat akan melakukan pencarian dokumen berdasarkan kata kunci yang telah disimpan sebelumnya. Pada aplikasi ini, seseorang juga dapat melakukan sharing dokumen kepada pihak lain. Sehingga setiap dokumen yang diunggah dapat di set bersifat private/public. \

Perangkat lunak akan dibuat menggunakan website dengan menggunakan framework laravel. Laravel merupaka suatu framework untuk membangun suatu website dnegan bahasa pemrograman PHP dengan konsep MVC (model view controller). Pada bagian antarmuka menggunakan library boostrap dan pada bagian basis data akan menggunakan postgreSQL.

\section{Rumusan Masalah}
Adapun rumusan masalah dari deskripsi yang telah dipaparkan adalah:
\begin{itemize}
	\item Bagaimana cara kerja Relational Data Base System Management untuk melakukan pencatatan dokumen?
	\item 
	\item Bagaimana membangun perangkat lunak untuk melakukan penambahan, penghapusan, pencarian sederhana terhadap struktur informasi dokumen yang telah dibuat?
	
\end{itemize}

\section{Tujuan}
Adapun tujuan penelitian dari rumusan masalah yang telah dipaparkan adalah:
\begin{itemize}
	\item Mempelajari cara kerja penyimpanan data biner dengan Relational Data Base Management System.
	\item Membangun struktur penyimpanan dokumen berbasis web yang memungkinkan untuk dapat dilakukan pencarian terkait kata kunci tertentu yang sebelumnya telah dimasukan oleh pengguna.
	\item Membangun sistem pencatatan dokumen berbasi web, yang dapat menyimpan beberapa jenis dokumen (word, excel, pdf, jpg, dll).
	\item Membuat Aplikasi yang dapat melakukan penambahan, penghapusan, pencarian sederhana terhadap struktur informasi yang telah dibuat.
	
\end{itemize}

\section{Deskripsi Perangkat Lunak}
Tuliskan deksripsi dari perangkat lunak yang akan anda hasilkan. Apa saja fitur yang disediakan oleh PL tersebut dan apa saja kemampuan dari PL tersebut. Perhatikan contoh di bawah ini:

Perangkat lunak akhir yang akan dibuat memiliki fitur minimal sebagai berikut:
\begin{itemize}
	\item Pengguna dapat menyimpan beberapa jenis dokumen berupa word, excel, pdf, jpg, dll.
	\item Pengguna dapat mencari dokumen berdasarkan kata kunci yang telah disimpan sebelumnya.
	\item pengguna dapat melakukan sharing dokumen kepada pihak lain. Sehingga setiap dokumen yang diunggah dapat di set bersifat private/public
	
		
\end{itemize}

\section{Detail Pengerjaan Skripsi}
Tuliskan bagian-bagian pengerjaan skripsi secara detail. Bagian pekerjaan tersebut mencakup awal hingga akhir skripsi, termasuk di dalamnya pengerjaan dokumentasi skripsi, pengujian, survei, dll.

Bagian-bagian pekerjaan skripsi ini adalah sebagai berikut :
	\begin{enumerate}
		\item Melakukan studi literatur mengenai Relational Database Management System.
		\item Melakukan studi literatur mengenai Document Management Techniques \& Technology 
		\item Mempelajari php dengan menggunakan framework Laravel
		\item Melakukan analisis masalah dan analisis kebutuhan perangkat lunak.
		
		\item Menulis dokumen skripsi
	\end{enumerate}

\section{Rencana Kerja}
Rincian capaian yang direncanakan di Skripsi 1 adalah sebagai berikut:
	\begin{enumerate}
		\item Melakukan studi literatur mengenai Relational Data Base Management System untuk melakukan pencatatan dokumen.
		\item Melakukan studi literatur mengenai 		Document Management Techniques \& Technology
		\item Memepelajari php dengan menggunakan framework Laravel.
		\item Melakukan analisis masalah dan analisis kebutuhan perangkat lunak.
		\item Menulis dokumen skripsi.
\end{enumerate}

Sedangkan yang akan diselesaikan di Skripsi 2 adalah sebagai berikut:
\begin{enumerate}
\item
\item
\item
\end{enumerate}

\vspace{1cm}
\centering Bandung, \tanggal\\
\vspace{2cm} \nama \\ 
\vspace{1cm}

Menyetujui, \\
\ifdefstring{\jumpemb}{2}{
\vspace{1.5cm}
\begin{centering} Menyetujui,\\ \end{centering} \vspace{0.75cm}
\begin{minipage}[b]{0.45\linewidth}
% \centering Bandung, \makebox[0.5cm]{\hrulefill}/\makebox[0.5cm]{\hrulefill}/2013 \\
\vspace{2cm} Nama: \makebox[3cm]{\hrulefill}\\ Pembimbing Utama
\end{minipage} \hspace{0.5cm}
\begin{minipage}[b]{0.45\linewidth}
% \centering Bandung, \makebox[0.5cm]{\hrulefill}/\makebox[0.5cm]{\hrulefill}/2013\\
\vspace{2cm} Nama: \makebox[3cm]{\hrulefill}\\ Pembimbing Pendamping
\end{minipage}
\vspace{0.5cm}
}{
% \centering Bandung, \makebox[0.5cm]{\hrulefill}/\makebox[0.5cm]{\hrulefill}/2013\\
\vspace{2cm} Nama: \makebox[3cm]{\hrulefill}\\ Pembimbing Tunggal
}
\end{document}

